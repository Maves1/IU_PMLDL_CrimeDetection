\documentclass[12pt]{article}

\usepackage{physics} % provides lots of nice features and commands often used in physics, it also loads some other packages (like AMSmath)
\usepackage{siunitx} % typesets numbers with units very nicely
\usepackage{enumerate} % allows us to customize our lists

\begin{document}

\title{Joke Generator}
\author{Timur Aizatvafin, Vsevolod Mikulik, Andrey Starodumov}

\maketitle

\begin{abstract}
	This report describes our attempts at joke generation. Briefly speaking, we tried both fine-tuning a pre-trained GPT-2 and
    training a GPT from scratch (with resources available, of course).
\end{abstract}

\section{Fine-tuning a pre-trained GPT}
	TODO
    
\subsection{Data Preprocessing}

\subsection{Training}

\subsection{Results}

\section{Training a GPT from Scratch}

We also came up with an idea of training a GPT from scratch. We found an implementation [TODO: parameters] and trained it on the dataset
of jokes from Reddit [TODO: link].

\subsection{Data Preprocessing}

\subsection{Training}

\subsection{Results}

\section{Conclusion}



\end{document}